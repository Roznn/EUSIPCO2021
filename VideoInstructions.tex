\documentclass[english,a4paper,11pt,oneside,onecolumn]{article}
\usepackage[utf8]{inputenc}
 \usepackage[pdftex]{hyperref}                  
 \hypersetup{backref,bookmarks=true,pdfpagemode=Fullscreen,linkcolor=black,colorlinks=true,urlcolor=blue}


\setlength{\topmargin}{-1cm}
\setlength{\headsep}{.5cm}
%\setlength{\footskip}{1.0cm}
\setlength{\textheight}{24.7cm}
\setlength{\textwidth}{17cm}
\setlength{\evensidemargin}{-.5cm}
\setlength{\oddsidemargin}{-.5cm}
\renewcommand{\baselinestretch}{1.5}

\title{Author Video Instructions}
\author{
\url{https://eusipco2021.org}\\
Twitter: \href{https://twitter.com/eusipco2021}{ @eusipco2021}\\
LinkedIn: \href{https://www.linkedin.com/company/eusipco/}{https://www.linkedin.com/company/eusipco/}
}
\date{June 2021}

\begin{document}

\maketitle

\section{Introduction}
In view of the COVID-19 pandemic, EUSIPCO 2021 has been moved to a fully virtual conference. We have opted for a format where each contributed paper presentation will take the form of a pre-recorded video, available during the original dates of the conference and on-demand after the conference concludes.
\textbf{For the live part of the conference, you will be asked to participate in live Q\&A with other authors and the  chair of your session. }

We  request you to prepare and upload a video of your presentation, comprised of a brief introduction with webcam (if desired), followed by voice over slides for your presentation. This will be posted within your presentation time slot on the EUSIPCO 2021 on-demand virtual platform. Please note that the file must be a video file in MP4 format (more details below). 

\section{Instructions for recording your presentation}

There are several video conferencing tools available to easily record a presentation. You may use any recording software as long as you obtain a high-quality recording and your final file is in the MP4 format. Here are some links to instructions on recording a meeting on common platforms:
\begin{itemize}
    \item You can use the two-step method  with Powerpoint: \href{http://support.office.com/en-us/article/record-a-slide-show-with-narration-and-slide-timings-0b9502c6-5f6c-40ae-b1e7-e47d8741161c}{Create Voice Over Power point} and \href{https://nursing.vanderbilt.edu/knowledge-base/knowledgebase/how-to-save-voppt-to-mp4/}{convert to MP4} 
    \item You can use video editors such as the   
    \href{https://support.microsoft.com/en-us/windows/create-or-edit-video-in-windows-10-53b3e8f8-a85f-172f-4efd-2e66afccf43e}{one provided in Windows 10} or the  \href{https://shotcut.org}{free, open source, cross-platform video editor shotcut.org} to combine the recorded audio with visuals.
\end{itemize}
Alternatively many video conferencing also provide recording and screen capture functionalities: 
\begin{itemize}
    \item Use Zoom: \href{https://support.zoom.us/hc/en-us/articles/201362473-Local-Recording}{Local Recording – Zoom Help Center}
    \item WebEx: \href{https://help.webex.com/en-us/n62735y/Record-a-Cisco-Webex-Meeting}{Video Conferencing - Record a Cisco Webex     Meeting   }
    \item Skype \href{https://www.bemidjistate.edu/offices/its/knowledge-base/skype-for-business-recording-a-meeting/}{Skype for Business: Recording a Meeting | Information Technology Services | Bemidji State University}
    \item Google Meet \href{https://support.google.com/meet/answer/9308681?hl=en}{Record a video meeting - Meet Help}
    \item Gotomeeting \href{https://www.techwalla.com/articles/how-to-record-a-gotomeeting-session}{How to Record a GoToMeeting Session | Techwalla}
    \item Microsoft Teams \href{https://support.office.com/en-us/article/record-a-meeting-in-teams-34dfbe7f-b07d-4a27-b4c6-de62f1348c24}{Record a meeting in Teams - Office Support}
    \item The free software OBS allows also to capture simultaneously yourself while speaking and your slides: 
\href{https://obsproject.com}{obsproject.com}

\end{itemize}

\section{Video \& caption Formats}

\paragraph{Accessibility:} Please note that you are also  encouraged to provide closed captions for your video, either in a \texttt{*.SRT} file, a \texttt{*.VTT} file. Information on how to obtain captions for your video recording can be found here:\\
\footnotesize{\href{https://www.rev.com/blog/resources/zoom-closed-captioning-how-to-caption-subtitle-zoom-meetings-and-recordings}{https://www.rev.com/blog/resources/zoom-closed-captioning-how-to-caption-subtitle-zoom-meetings-and-recordings}}. 
\normalsize

\paragraph{Captions, Audio \& visuals:} A good way to create a video  is in starting to write the script (the spoken text that becomes the captions) and then record the speech read from that script (corresponding to the audio track of the video). Reading from a script is about 1.5 times faster than a standard live slide presentation\footnote{\url{https://en.wikipedia.org/wiki/Words_per_minute}} and it is expected that the video will be shorter than the maximum duration of 15 minutes\footnote{See recommendations at \url{https://eusipco2021.org/virtual-participation/}} without any difficulty. The visuals (i.e. slides) can then be synchronised with the audio track using a video editor. 
It is recommended to have slide numbers in the presentation. 


\paragraph{Guidelines for preparing your video \& captions:} 
\begin{itemize}
    \item Maximum Duration: 15 minutes 
    \item Maximum File size: 250MB 
    \item Video file format: MP4
    \item Closed captions for your video, either in a .SRT file or a .VTT file
    \item Dimensions: Minimum height 720 pixels, aspect ratio: 16:9
\end{itemize}

Additionally:
\begin{itemize}
    \item Please note the final specifications will be checked at the time of submission and files not compliant may not be uploaded.

\item Please be sure the video includes the title of the paper, the authors, and a mention to EUSIPCO 2021.
\end{itemize}

\paragraph{PowerPoint template \& conference logo.} To help presenters, a promotional pack with  a PowerPoint template and EUSIPCO 2021 logo has been made available on the conference website at \url{https://eusipco2021.org/promote-eusipco-2021/} .

\section{Deadline for video submission: June 30, 2021}

In order for videos to be verified by the technical program committee, there is considerable urgency in preparing and uploading your video. Accordingly, we ask you that you please finalize your video no later than \underline{June 30, 2021}. \textbf{This is a HARD DEADLINE. Presentations will not be accepted after this date.}

\section{Speaker (contact author) Instructions}

\begin{enumerate}
    \item 
	Visit the paper management system at the following URL:\\ \url{https://cmsworkshops.com/EUSIPCO2021/Papers.asp}
\item	Login with your paper number and password provided in the paper submission confirmation email.
\item 	Choose which author is the presenting author and click on the "Continue to Video Upload" button.
\item 	You will be redirected to a “Speaker Submission” form. %at \url{XXX}

\item	Fill out the form
\begin{enumerate}
    \item A series of fields will be pre-filled on the form (these will be locked for editing). Please double check these for accuracy. If there are any inaccuracies, please contact:\\  \href{mailto:cfoster@conferencecatalysts.com}{cfoster@conferencecatalysts.com} and reference your paper number. 
    \item 	Upload your presentation video in mp4 format (max of 250MB is allowed)
\newpage
    \item 	\underline{Optional:}
\begin{enumerate}
    \item Upload a Speaker headshot photo
\item	Include a Speaker Biography in the available text field
\item	Upload a PDF of your presentation slides %(Only include if you are collecting slides)
%\item	Upload a PDF Poster (Only include if you are collecting posters)
\end{enumerate}
\end{enumerate}
\item Submit the form by selecting the Save button at the bottom of the screen
\item 	A confirmation message will be presented upon successful submission and a confirmation will be emailed to you
\begin{enumerate}
\item 	Be sure to retain this information in case you need to make future updates to your submission
\end{enumerate}
\end{enumerate}

\section{Help \& concluding remarks }

\paragraph{Help:} If you have any issues with uploading your presentation, please contact: \\
\href{mailto:cfoster@conferencecatalysts.com}{cfoster@conferencecatalysts.com}.

\paragraph{No show policy:} As a reminder, videos that are not received by the deadline will be considered no-shows in accordance with the non-presented paper policy.

\paragraph{Thank you (Go raibh maith agat)!} We hope to make EUSIPCO 2021 a success despite the challenges created by the COVID-19 pandemic. We thank you for your cooperation in this endeavor.



\end{document}
